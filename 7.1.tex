\documentclass[10pt]{article}
\usepackage{amsmath}
\usepackage{amsfonts}
\usepackage[utf8x]{inputenc}
\usepackage[russian]{babel}
\begin{document}
\begin{flushleft}
\bf{7.1 \quad Замкнутые системы булевых функций}
\end{flushleft}
Пусть R — произвольное множество булевых функций. \textit{Замыканием} множества $R$
называется множество всех функций, которые можно реализовать
формулами в базисе $R$. Замыкание множества $R$ будем обозначать через [$R$].
Множество булевых функций $R$ называется \textit{(функционально) замкнутым}
множеством, если оно совпадает со своим замыканием, т. е. $R$ = [$R$]. \par
Рассмотрим пять важнейших замкнутых множеств в $P_2$. Часто рассматриваемые ниже замкнутые
множества называются также замкнутыми классами.\par
{\bf{1.}} Будем говорить, что функция $f(x_1,\dots,x_n)$ сохраняет нуль, если
\begin{center} $f(0,\dots, 0) = 0.$ \end{center}
Множество, состоящее из всех булевых функций сохраняющих нуль, обозначается
через $T_0$. Легко видеть, что функции $0$, $x$, $x\& y$, $x \vee y$ и $x \otimes y$
принадлежат $T_0$, а функции $1$, $\bar x$, $x \sim y$, $x \mid y$, $x \downarrow y$ и $x \to y$
 не принадлежат $T_0$. \par
Так как тождественная функция сохраняет нуль, и для любых сохраня-
ющих нуль функций ${f_0, f_1, \dots,f_k}$ справедливо равенство
\begin{center} $f(0, \dots, 0) = f_0(f_1(0, \dots, 0), \dots,f_k(0, \dots, 0)) = f_0(0, \dots, 0) = 0.$\end{center}
т. е. реализуемая формулой $f_0(f_1, \dots,f_k)$ функция $f$ также сохраняет нуль,
то легко видеть, что множество $T_0$ замкнуто.
Любой булев вектор длины $2^n$ с первой нулевой компонентой будет вектором
значений функции из $T_0$. Поэтому, в T0 содержится ровно $2^{2^n−1}$ функций
из $P_{2}(n)$. Множество $T_0 \cap P_{2}(n)$ будем обозначать через $T_{0}(n)$. \par
{\bf{2.}} Будем говорить, что функция $f(x_1, \dots,x_n)$ сохраняет единицу, если
\begin{center} $f(1, \dots, 1) = 1.$\end{center}
Множество, состоящее из всех булевых функций сохраняющих единицу,
обозначается через $T_1$.
\newpage \par
Легко видеть, что функции $1$, $x$, $x\&y$, $x \vee y$, $x \sim y$ и $x \to y$ принадлежат
$T_1$, а функции $0$, $\bar x$, $x \otimes y$, $x \mid y$ и $x \downarrow y$ не принадлежат $T_1$. Доказательство
замкнутости множества $T_1$ аналогично доказательству замкнутости множества
$T_0$. Также легко видеть, что в $T_1$ содержится ровно $2^{2^n−1}$ функций из
$P_{2}(n)$. Множество $T_1 \cap P_{2}(n)$ будем обозначать через $T_{1}(n)$. \par
{\bf{3.}}\, Будем говорить, что булева функция $f(x_1,\dots,x_n$) является \textit{двойственной} к функции $g(x_1, \dots, x_n$), если
\begin{center} $f(x_1, \dots, x_n) = \bar{g}(\bar{x}_1, \dots, \bar{x}_n)$\end{center}
Функцию двойственную к функции $f$ будем обозначать через $f^{*}$. Легко видеть,
 что $(f^{*})^{*} = f$ для любой булевой функции $f$. Из законов двойственности
 следует, что $(x\&y)^{∗} = x \vee y$ и $(x \vee y)^{*} = x\&y.$ Функция $f$ называется
\textit{самодвойственной}, если $f = f^{*}$. Множество, состоящее из всех самодвойственных
 булевых функций, обозначается через $S$. Самодвойственными являются
 функции $x$, $\bar{x}$, $x_1 \cap x_2 \cap x_3.$ Среди булевых функций, существенно
  зависящих ровно от двух переменных, нет ни одной самодвойственной
функции. \par
Докажем замкнутость множества самодвойственных функций. Пусть
$f_0, f_1, \dots, f_k$ — произвольные самодвойственные функции. Рассмотрим новую
функцию $f = f_0(f_1, \dots, f_k)$. Так как добавление фиктивной переменной
оставляет самодвойственную функцию самодвойственной, то без ограничения
общности будем полагать, что все функции $f_i$ зависят от одних и тех
же переменных $x_1,\dots, x_n$. Тогда\\
\begin{align*}
f(\bar{x}_1, \dots, \bar{x}_n) &= f_0(f_1(\bar{x}_1, \dots, \bar{x}_n), \dots, f_k(\bar{x}_1, \dots, \bar{x}_n))= \\
                               &= f_0(\bar{f}_1(x_1, \dots, x_n), \dots, \bar{f}_k(x_1, \dots, x_n))= \\
                               &= \bar{f}_0(f_1(x_1, \dots, x_n), \dots, f_k(x_1, \dots, x_n)) = \bar{f}(x_1, \dots, x_n).
\end{align*}
Следовательно, функция $f$ — самодвойственная. Таким образом, множество
$S$ замкнуто. \par
Так как каждая самодвойственная функция на противоположных наборах
принимает противоположные значения, то для определения любой
самодвойственной функции достаточно задать ее значения только на половине
из $2^n$ наборов. Следовательно, в $S$ содержится ровно $2^{2^n−1}$ функций
из $P_2(n)$. Далее множество самодвойственных функций, зависящих от n
переменных, будем обозначать через $S(n)$.\par
{\bf{4.}}\, Функция $f(x_1, \dots,x_n)$ называется линейной, если степень ее многочлена
Жегалкина не превосходит единицу,~т.~е.
\begin{center} $f(x_1, \dots, x_n) = \alpha_1 x_1 \otimes \dots \otimes \alpha_n x_n \otimes \alpha_0$,\end{center}
где $\alpha_i$ — булевы постоянные. Множество, состоящее из всех линейных булевых
функций, обозначается через $L$. Очевидно, что среди функций из $P_2(2)$
линейными являются только $0$, $1$, $x$, $\bar{x}$, $y$, $\bar{y}$, $x \otimes y$ и $x \sim y$. Непосредственно
из определения линейной функции следует замкнутость множества $L$.\par
Так как каждая булева функция однозначно определяется коэффициентами
своего многочлена Жегалкина, а укаждой линейной функции все
коэффициенты при одночленах степени два и выше равны нулю, то легко
видеть, что в $L$ содержится ровно $2^{n+1}$ функций из $P_2(n)$.
Далее множество $L \cap P_2(n)$ будем обозначать через $L(n)$.\par
{\bf{5.}}\,Функция $f(x_1, \dots, x_n)$ называется \textit{монотонной}, если
\begin{center} $f(\alpha_1, \dots, \alpha_n) \le f(\beta_1, \dots, \beta_n)$\end{center}
для любых наборов ${\bf{a}}=(\alpha_1, \dots, \alpha_n)$ и ${\bf{b}}=(\beta_1, \dots, \beta_n)$ таких,
что ${\bf a}\preceq{\bf b}$. Множество, состоящее из всех монотонных булевых функций, обозначается через $M$.
В $P_2(2)$ монотонными являются функции 0, $1$, $x$, $y$, $x \& y$ и $x \vee y$.
Докажем замкнутость множества монотонных функций. Пусть $f_0, \dots, f_k$
— произвольные монотонные функции. Очевидно, что добавление фиктивной
переменной оставляет монотонную функцию монотонной. Поэтому без
ограничения общности будем полагать, что все функции $f_i$ зависят от одних
и тех же переменных $x_1, \dots, x_n$. Пусть {\bf{a}}, {\bf{b}} — такие наборы из $\mathbb{B}^n$, что ${\bf{a}} \preceq {\bf{b}}$.
Рассмотрим новую функцию $f=f_0(f_1, \dots, f_k)$. Так как $f_i({\bf{a}}) \le f_i({\bf{b}})$, то $(f_1({\bf{a}}), \dots, f_k({\bf{a}})) \preceq (f_1({\bf{b}}), \dots, f_k({\bf{b}}))$, и поэтому
\begin{center} $f({\bf{a}})=f_0(f_1({\bf{a}}), \dots, f_k({\bf{a}})) \le f_0(f_1({\bf{b}}), \dots, f_k({\bf{b}})) = f({\bf{b}})$.\end{center}
Следовательно, функция $f$ — монотонная. Таким образом, множество $M$ замкнуто.\par
Обозначим множество монотонных функций $n$ переменных через $M(n)$. В отличие от множеств $T_0(n),
T_1(n), S(n)$ и $L(n)$, мощности которых легко были найдены выше, точное число монотонных
функций в $P_2(n)$ при больших $n$ неизвестно. Лишь сравнительно недавно А. Д. Коршуновым была
найдена асимптотически точная формула для $|M(n)|$. Эта формула выглядит достаточно громоздко,
и поэтомуздесь не приводится. Вместо этого
для $|M(n)|$ ниже устанавливаются более грубые, но в тоже время значительно более простые неравенства.
\end{document}
